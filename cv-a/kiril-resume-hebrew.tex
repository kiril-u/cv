\setRTL
\hspace{.25\textwidth}
%%% - TITLE (NAME) IN HEBREW - %%%
\begin{minipage}[t]{.5\textwidth}
	\par{\centering{\Huge  \textbf{אוריבסקי קיריל}}\par}
\end{minipage}
\section{כללי}
\unsetRTL
\setLTR
\begin{minipage}[t]{.5\linewidth}
	\begin{tabular}{rp{.75\linewidth}}
		\baselineskip=20pt
		\email{} : &\href{mailto:kirilurivsky@gmail.com}{kirilurivsky@gmail.com}\\
		\phone{} : &\href{tel:972525470608}{(+972) 52 547 0608}
	\end{tabular}
\end{minipage}
\begin{minipage}[t]{.5\linewidth}
	\begin{tabular}{rl}
		\linkedin{} : &\href{https://www.linkedin.com/in/kiril-u}{https://linkedin.com/in/kiril-u}\\
		\location{} : &\href{https://goo.gl/maps/MSacjpSy7vZSKykP7}{דרום מחוז אשדוד,}
	\end{tabular}
\end{minipage}	
\unsetLTR
\unsetRTL
\setRTL
\begin{minipage}[t]{1\linewidth}
	%\begin{tabular}{rl}
	\vspace{0.25cm} %%%%%%%%%5 - Brief self description
	סטודנט לתואר מוסמך בפסיכולוגיה חינוכית המאופיין בנטייה לרכוש ידע באופן עצמאי כנגזרת מקיומה של משיכה ועניין עבור מגוון רחב של תחומי ידע ואומנויות.  שילובם של אלו הביא לפיתוח היכולת לאתר מידע רלוונטי בזריזות, לעבדו ביעילות, והעברתו הלאה באופן שהינו קוהרנטי וחסכוני. [פסקה זו משתנה בהתאם למשרה].
	%\end{tabular}
\end{minipage}

\section{כישורים ויכולות רלוונטיות}

\begin{itemize}
\item{ פה תהיינה}
\item{ רשימה שאכתוב}
\item{ בהתאם לדרישות המשרה}
\item{ ראו דוגמא לכך,}
\item{ בהמשך הקובץ בגרסתו האנגלית}
\item{ בה כתבתי את חלק זה עבור משרת "עוזר אקדמי"}
\end{itemize}

\section{השכלה}

\entry{2019}
{\textbf{תואר מוסמך (מ.א.) בפסיכולוגיה חינוכית (מתקיים); המכללה האקדמית עמק יזרעאל ע"ש מקס שטרן, עמק יזרעאל}}
{במסגרת הכשרתי המעשית (הפרקטיקום) לקחתי חלק פעיל בישיבות ייעוץ אשר התקיימו עבור סגל ההוראה וחברי צוות ההנהלה של בית ספר תיכון וגן חינוך מיוחד בעפולה. בנוסף לכך, נכחתי בוועדות השמה ובוועדות זכאות ואפיון עבור ילדים ובני נוער אשר הופנו לשירות הפסיכולוגי החינוכי מטעמו  ערכתי אבחונים פסיכודיאגנוסטיים ועל סמך ממצאיהם כתבתי דוחות עבור הכתובות בהתאם לצרכיי המטופל, כך שחלקו מאלו הופנו למשרד החינוך ולמשרדי הרווחה. 
	במקביל לכך, אני עורך מחקר על פיו נכתבת עבודת תזה הנכתבת תחת ליוויו והדרכתו של 
	\href{https://www.researchgate.net/scientific-contributions/Orrie-Dan-2172412162}{ד"ר דן אורי} אודות 
	\href{https://tinyurl.com/ADHD-Coping-Seminar}{השפעותיהן של הפרעות קשב והיפראקטיביות ולקויות למידה לאורך חייהם ועל פני מספר תחומי תפקוד התפתחותי.}}
{}

\entry{2015--2018}
{\href{https://www.dropbox.com/s/pcm0mybvgi85ma0/BA-Psych.pdf}{תואר ראשון בפסיכולוגיה ובלימודים רב תחומיים במדעי הרוח והחברה. סיום התואר \textbf{בהצטיינות יתרה}}, ממוצע ציונים: 95.00.}
{בעל תעודת הצטיינות (2017) ותעודת הצטיינות רקטור (2018) שניתנה בשל הישגיי הלימודיים.}
{לימודיי הרב תחומיים במדעי הרוח והחברה כללו את החטיבות לפילוסופיה ולניהול ויישוב סכסוכים ומו"מ}

\subsection{הכשרות נוספות}
\entry{2009--2010}
 {קורס
\href{https://www.dropbox.com/s/csphxdynjdg3ze2/computer-technitian-certificate.jpg}{טכנאי מחשבים}
		\textbf{, IQline, האקדמיה למדע ואומנות}}
{ביצוע הליכי אבחון, תיקון; וטיפול ברכיבי תוכנה וחומרה עבור שימור המשכיות עבודתן התקינה של מערכות מחשוב. \\
בשל סקרנות ועניין בעיסוקים שונים במחשבים מגיל צעיר פיתחתי אוריינטציה טכנולוגית שיכולתי ליישם לטובתי גם בתחומים אחרים.}
{}
{}

\section{ניסיון תעסוקתי ושירות צבאי}

\entry{2018--2019}
{\textbf{מדריך שיקומי}  בהוסטל לנפגעי נפש "בית כנרת", קידום פרוייקטים שיקומיים}
{ליווי והדרכה של משתקמים בניהול אורח חיים עצמאי ושילוב בקהילה על ידי בנייתה ויישומה של תוכנית שיקום פרטנית בעזרתו של צוות רב מקצועי. פרט לעבודה השגרתית עם כלל דיירי הקהילה, ביצעתי ליווי פרטני מורחב עבור חמישה משתקמים איתם הייתי מקיים 2-4 מפגשים מדי שבוע.}
{}

\entry{2013--2015}
{\href{https://www.dropbox.com/s/kr5rcui1zgp35i0/recommendation-letter-security-guard.jpg?dl=0}{\textbf{מאבטח מתקנים} , רשף בטחון}} 
{2015 - משרדי בנק לאומי בראשון לציון ובלוד. 
	\\		
	2013 - מרכזי לוגיסטיקה של חברת Cargo Flying באזור תעשייה כנות.}
{}

\entry{2013--2014} 
{\textbf{מש"ק נפגעים בחיל ההנדסה הקרבית} , מפקדת זרוע היבשה}
{סיוע בטיפול בצורכיהם של חיילים מאושפזים וליווי חיילים הנמצאים בחופשת מחלה בביתם. כמו כן, מתן תמיכה עבור משפחות שכולות.}
{}

\section{שפות}

\entry{טבעיות}
{\textbf{אנגלית, עברית ורוסית}}
{בעל כישוריי שפה מפותחים וידע לשוני נרחב בשלושת השפות הנקובות מפאת הצורך התכוף בשימושם. לאור זאת התפתחה היכולת לתקשר רעיונות ומושגים מופשטים בצורה ברורה ויעילה יחד עם הבנה פרגמטית מעמיקה באשר לאופן השימוש הראוי בכל אחת מהשפות בהתאם להקשר הנתון ועל סמך עקרונות אתיים כמו רגישות תרבותית ונימוס.} 
{}

\entry{מחשב}
{\textbf{שפות מחשבים}}
{- באש (bash) ושפות מעטפת ותפעול (shell) תואמות POSIX דומות המיוחסות לפרוייקט GNU ומערכות הפעלה מבוססות UNIX \\ 
	- שפות סימון ותבנות כמו HTML, XML, Markdown, ו-
	\textit{LaTex}/\textit{XeTeX} \\
	- הבנה בסיסית בכתיבה בשפות תכנות Java, Javascript ו-C.}
{}

\section{התנדבות ועשייה למען הקהילה}

\entry{2017--2018}
{\textbf{התלמדות טיפולית במרכז בריאות הנפש}, מרפאת הדקל, אילת}
{מתן הכוונה, ליווי, תמיכה רגשית, וסיוע בפיתוח מסוגלות עצמית בקרב ילדים ובני נוער המשויכים למרפאה}
{}

\entry{2015--2018}
{\textbf{חונך אישי עבור ילדים ונערים בסיכון במסגרת פרויקט פר"ח}}
{השתתפות רצופה לאורך שלוש שנים}
{}		
\unsetRTL